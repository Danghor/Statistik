\section{Praktische Berechnung von Fakult�t und \\ Binomial-Koeffizienten}
In der Wahrscheinlichkeits-Rechnung verfolgen uns die Fakult�t und die 
Binomial-Koeffizienten auf Schritt und Tritt.  F�r kleine Werte von $n$ k�nnen wir $n!$
und ${n \choose k}$ problemlos mit dem Taschenrechner ausrechnen.  Aber schon bei der
L�sung Aufgabe 5 st��t der Taschenrechner an seine Grenzen, denn der Ausdruck 
\\[0.1cm]
\hspace*{1.3cm}
${40\,000 \choose 200}$
\\[0.1cm]
der bei der L�sung dieser Aufgabe im Nenner auftritt, liefert eine ganze Zahl mit 545
Stellen.  Solche Zahlen sind mit einem gew�hnlichen Taschenrechner nicht mehr
darstellbar.   Wir entwickeln daher in diesem Abschnitt eine N�herungs-Formel f�r den
Binomial-Koeffizienten. 

Damit dies m�glich ist, m�ssen wir zun�chst eine N�herung f�r die Fakult�t angeben. 
Die klassische N�herung von Stirling (James Stirling; 1692 - 1770) lautet 
\\[0.1cm]
\hspace*{1.3cm}
$\displaystyle n! \approx \sqrt{2 \cdot \pi \cdot n} \cdot \frac{n^n}{e^n}$.
\\[0.1cm]
Genauer ist die Formel von Lanzcos (Cornelius Lanzcos; 1893 - 1974), sie lautet: 
\\[0.1cm]
\hspace*{1.3cm}
$\displaystyle n! \approx \sqrt{2 \cdot \pi\;} \cdot \left(\frac{n+ \frac{1}{2}}{e}\right)^{n+\frac{1}{2}}$.
\\[0.1cm]
Die Formel auf der rechten Seite approximiert die Fakult�t in dem folgenden Sinne: Es gilt
\\[0.1cm]
\hspace*{1.3cm}
$\displaystyle \lim\limits_{n\rightarrow \infty} \frac{\sqrt{2 \cdot \pi\;} \cdot
  \left(\frac{n+ \frac{1}{2}}{e}\right)^{n+\frac{1}{2}}}{n!} = 1$.
\\[0.1cm]
F�r eine Herleitung dieser Formeln bleibt uns leider nicht die Zeit.  Oft sind die
Fakult�ten so gro�, dass Sie nicht mehr auf dem Taschenrechner dargestellt werden k�nnen.
Sie treten dann meist in Br�chen auf und der zu berechnende Bruch ist durchaus noch
darstellbar. Dann kann es hilfreich sein, zum Logarithmus �berzugehen.   Um beispielsweise
${n \choose k}$ zu berechnen, gehen wir wie folgt vor: 
\begin{equation}
  \label{eq:binappr}  
\begin{array}[b]{lcl}
\displaystyle 
 {n \choose k} & = & \displaystyle \frac{n!}{k! \cdot (n-k)!} \\[0.3cm]
                & = & \displaystyle \exp\left(\ln\left(\frac{n!}{k! \cdot (n-k)!}\right)\right) \\[0.3cm]  
                & = & \displaystyle \exp\Bigl(\ln\bigl(n!\bigr) - \ln\bigl(k!\bigr) - \ln\bigl((n-k)!\bigr)\Bigr) \\[0.3cm]  
\end{array}
\end{equation}
Zu Berechnung der von $\ln(n!)$ wenden wir auf  die N�herungsformel von Lanzcos den
Logarithmus an und erhalten 
\\[0.1cm]
\hspace*{1.3cm}
$\displaystyle \ln\bigl(n!\bigr) \approx  \frac{1}{2}\cdot\ln(2\cdot \pi) + \left(n+\frac{1}{2}\right) \cdot \left( \ln\Bigl(n + \frac{1}{2}\Bigr) - 1 \right)$.
\\[0.1cm]
Setzen wir diesen Wert in Gleichung (\ref{eq:binappr}) ein, so  k�nnen wir den
Binomial-Koeffizienten approximieren.
Berechnen wir ${100 \choose 50}$ auf diese Weise, so erhalten wir 
\\[0.1cm]
\hspace*{1.3cm}
$\displaystyle {100 \choose 50} \approx 1.007667751 \cdot 10^{30}$.
\\[0.1cm]
Der relative Fehler liegt bei $1.2$\textperthousand.


Es sei $n = 2*m$ und $k = \frac{1}{2}n + s = m + s$. Also gilt $n - k = m - s$.
Die \emph{Stirling'sche Formel} lautet
\\[0.1cm]
\hspace*{1.3cm}
$\displaystyle 
  n! \approx \sqrt{2\pi n} \left(\frac{n}{e}\right)^n$.
\\[0.1cm] 
Logarithmieren dieser Formel liefert 
\\[0.1cm]
\hspace*{1.3cm}

$\displaystyle 
  \ln\bigl(n!\bigr) \approx \frac{1}{2} \ln(2\pi) + \frac{1}{2} \ln(n) + n\ln(n) - n$
\\[0.1cm]
Die Funktion $\ln(1+x)$ l�sst sich wie folgt in einer Taylor-Reihe entwickeln: 
\\[0.1cm]
\hspace*{1.3cm}
$\displaystyle \ln(1+x) \approx x - \frac{1}{2} x^2$, \qquad also 
$\displaystyle \ln(1-x) \approx - x - \frac{1}{2} x^2$.
\\[0.1cm]
Damit kann der Binomial-Koeffizient ${n \choose k}$ approximiert werden: 
\begin{equation}
  \label{eq:ml0}
  \begin{array}[t]{lcl}
\displaystyle
 \ln\Biggl( {n \choose k} \Biggr)
& = & \displaystyle \ln\left(\frac{n!}{k! (n-k)!}\right) \\[0.5cm]
& = & \displaystyle \ln\left(\frac{n!}{(m+s)! (m-s)!}\right) \\[0.5cm]
& = & \displaystyle \ln(n!) - \ln\bigl((m+s)!\bigr) - \ln\bigl((m-s)!\bigr) \\[0.3cm]
& \approx & \displaystyle 
            + \frac{1}{2} \ln(2\pi) + \frac{1}{2} \ln(n) + n\ln(n) - n \\[0.3cm]
&         & \displaystyle
            - \frac{1}{2} \ln(2\pi) - \frac{1}{2} \ln(m+s) - (m+s)\ln(m+s) + m + s \\[0.3cm]
&         & \displaystyle
            - \frac{1}{2} \ln(2\pi) - \frac{1}{2} \ln(m-s) - (m-s)\ln(m-s) + m - s \\[0.6cm]
& =       & \displaystyle 
            - \frac{1}{2} \ln(2\pi) + \frac{1}{2} \ln(n) + n\ln(n)  \\[0.3cm]
&         & \displaystyle
            - \frac{1}{2} \ln(m+s) - (m+s)\ln(m+s)  \\[0.3cm]
&         & \displaystyle
            - \frac{1}{2} \ln(m-s) - (m-s)\ln(m-s)  \\[0.6cm]
& =       & \displaystyle 
            - \frac{1}{2} \ln(2\pi) + \frac{1}{2} \ln(n) + n\ln(n)  \\[0.3cm]
&         & \displaystyle
            - \frac{1}{2} \ln(m) - \frac{1}{2}\ln\left(1+\frac{s}{m}\right) - (m+s)\ln(m) - (m+s)\ln\left(1+\frac{s}{m}\right)  \\[0.3cm]
&         & \displaystyle
            - \frac{1}{2} \ln(m) - \frac{1}{2}\ln\left(1-\frac{s}{m}\right) - (m-s)\ln(m) - (m-s)\ln\left(1-\frac{s}{m}\right)  \\[0.3cm]
\\[0.3cm]
  \end{array}
\end{equation}
Es gilt 
\\[0.1cm]
\hspace*{1.3cm}
$
\begin{array}[t]{lcl}
\displaystyle
 \frac{1}{2}\ln(n) - \frac{1}{2}\ln(m) - \frac{1}{2}\ln(m) & = &
\displaystyle
 \frac{1}{2}\ln(n) - \ln\bigl(\frac{n}{2}\bigr) \\[0.3cm]
& = & \displaystyle
   \frac{1}{2}\ln(n) - \ln(n) + \ln(2)  \\[0.3cm]
& = & - \frac{1}{2}\ln(n) + \ln(2)  \\[0.3cm]
\end{array}
$
\\[0.1cm]
Weiter gilt 
\\[0.1cm]
\hspace*{1.3cm}
$
\begin{array}[t]{lcl}
\displaystyle n\ln(n) - (m+s)\ln(m) - (m-s)\ln(m) & = & \displaystyle n\ln(n) - 2m\ln(m) \\[0.3cm] 
&=& \displaystyle n\ln(n) -  n\ln\left(\frac{n}{2}\right) \\[0.3cm] 
&=& \displaystyle n\ln(n) - n\ln(n) + n\ln(2) \\[0.3cm]
&=& \displaystyle n\ln(2)
\end{array}
$
\\[0.1cm]
Wir haben 
\\[0.1cm]
\hspace*{1.3cm}
$\displaystyle
  - \frac{1}{2}\ln\left(1+\frac{s}{m}\right) - \frac{1}{2}\ln\left(1-\frac{s}{m}\right) \approx
- \frac{1}{2} \left(\frac{s}{m} - \frac{1}{2}\frac{s^2}{m^2} - \frac{s}{m} - \frac{1}{2}\frac{s^2}{m^2}\right)
= \frac{1}{2}\frac{s^2}{m^2}
$
\\[0.1cm]
Es gilt 
\\[0.1cm]
\hspace*{1.3cm}
$
\begin{array}[t]{cl}
& \displaystyle 
(m+s)\ln\left(1+\frac{s}{m}\right)  + (m-s)\ln\left(1-\frac{s}{m}\right) \\[0.3cm]
\approx & \displaystyle
(m+s)\left(\frac{s}{m} - \frac{1}{2}\frac{s^2}{m^2}\right)  + (m-s)\left(-\frac{s}{m} - \frac{1}{2}\frac{s^2}{m^2}\right) \\[0.6cm]
\approx & \displaystyle
\left(s - \frac{1}{2}\frac{s^2}{m}\right)  + \left(-s - \frac{1}{2}\frac{s^2}{m}\right) +
\left(\frac{s^2}{m} - \frac{1}{2}\frac{s^3}{m^2}\right) + \left(\frac{s^2}{m} + \frac{1}{2}\frac{s^3}{m^2}\right) \\[0.6cm]
= & \displaystyle
\frac{s^2}{m}
\end{array}
$
\\[0.1cm]
Damit ergibt sich
\\[0.1cm]
\hspace*{1.3cm}
$
\begin{array}[t]{cl}
& \displaystyle
  \ln\Biggl( {n \choose k} \Biggr) \\[0.6cm]
\approx & \displaystyle
- \frac{1}{2}\ln(2\pi) - \frac{1}{2}\ln(n) + \ln(2) + n\ln(2) + \frac{1}{2}\frac{s^2}{m^2} - \frac{s^2}{m}  \\[0.6cm]
\approx & \displaystyle
 \ln\left(\sqrt{\frac{2}{\pi n}}\right) + \ln\bigl(2^n\bigr) - \frac{s^2}{\frac{n}{2}}  \\[0.6cm]
= & \displaystyle
 \ln\left(\sqrt{\frac{2}{\pi n}}\right) + \ln\bigl(2^n\bigr) - \frac{\bigl(k - \frac{1}{2}n\bigr)^2}{\frac{n}{2}},  \\[0.6cm]
\end{array}
$
\\[0.1cm]
wobei wir im zweiten Schritt den Term $\frac{1}{2}\frac{s^2}{m^2}$ gegen�ber den anderen Termen vernachl�ssigt haben.
Damit haben wir die folgende Formel hergeleitet:
\\[0.3cm]
\hspace*{1.3cm}
$\displaystyle
 {n \choose k} \approx \sqrt{\frac{2}{\pi n}} \cdot 2^n \cdot \exp\left(-\frac{\bigl(k - \frac{1}{2}n\bigr)^2}{\frac{1}{2}n}\right)$
\\[0.1cm]


%%% Local Variables: 
%%% mode: latex
%%% TeX-master: "statistik"
%%% End: 
